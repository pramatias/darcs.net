\chapter{Repository formats}

Darcs 2 supports three repository formats:

\begin{itemize}
\item The current format, called `darcs-2'.  It has the most features.
\item The original format, called `darcs-1' or `old-fashioned inventory'.
  It is the only format supported by Darcs releases prior to 2.0.
  Patches cannot be shared between darcs-2 and darcs-1 repositories.
\item An intermediary format, called `hashed'.  It provides some of
  the features of darcs-2, while retaining the ability to exchange
  patches with darcs-1.  Patches cannot be shared between darcs-2 and
  hashed repositories.
\end{itemize}

Note that references to the hashed format refer to hashed darcs-1
repositories.  All darcs-2 repositories are also hashed, and `darcs
show repo' in a darcs-2 repository will report the format as `hashed,
darcs-2'.

The hashed (and darcs-2) format improves on the darcs-1 format in the
following ways:

\begin{itemize}
\item Improved atomicity of operations.  This improves the safety and
  efficiency of those operations.
\item File names in the `pristine' copy of the working tree are
  hashed.  This greatly reduces the chance of a program accidentally
  treating files in \_darcs/pristine as part of the working tree.

  For example, in darcs-1 running \verb|find -name "*.gif" -delete|
  instead of \verb|find -name _darcs -prune -o -name "*.gif" -delete|
  would result in GIF files in pristine files being deleted.

\item Support for `lazy' repositories.  This causes patches to be
  copied from the parent repository on demand, rather than during the
  initial `darcs get'.  For repositories that only operate on recent
  patches (such as feature branches), this reduces the repository's
  size, and the time and bandwidth taken to create it.

  Lazy repositories are first-class repositories, and all operations
  are supported as long as the parent repository remains accessible.
  This isn't the case for the `partial' feature of the darcs-1 format.

\item Support for caches that are shared by all repositories.  When
  operating on many similar repositories, the patches and pristine
  files they have in common are hard links to the cache.  This greatly
  reduces storage requirements, and the time and bandwidth needed to
  make similar repositories.
\end{itemize}

The darcs-2 format improves on the hashed format in the following
ways:

\begin{itemize}
\item The `exponential merge' problem is \emph{far} less likely to
  occur (it can still be produced in deliberately pathological cases).
\item Identical primitive changes no longer conflict.  For example, if
  two patches both attempt to add a directory `tests', these patches
  will not conflict.
\end{itemize}

% The following reference is particularly important because init
% explains *when* to use each format, which I don't want to
% copy-and-paste here because then the two copies will get out of
% sync.

See also `darcs initialize' (\ref{initialize}), `darcs convert'
(\ref{convert}) and `darcs get' (\ref{get}).

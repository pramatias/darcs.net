%  Copyright (C) 2002-2003 David Roundy
%
%  This program is free software; you can redistribute it and/or modify
%  it under the terms of the GNU General Public License as published by
%  the Free Software Foundation; either version 2, or (at your option)
%  any later version.
%
%  This program is distributed in the hope that it will be useful,
%  but WITHOUT ANY WARRANTY; without even the implied warranty of
%  MERCHANTABILITY or FITNESS FOR A PARTICULAR PURPOSE.  See the
%  GNU General Public License for more details.
%
%  You should have received a copy of the GNU General Public License
%  along with this program; see the file COPYING.  If not, write to
%  the Free Software Foundation, Inc., 51 Franklin Street, Fifth Floor,
%  Boston, MA 02110-1301, USA.

\documentclass{book}
%\usepackage{color}

\usepackage{verbatim}
\usepackage{html}
\usepackage{fancyvrb}
\newenvironment{code}{\comment}{\endcomment}
% \newenvironment{code}{\color{blue}\verbatim}{\endverbatim}

\begin{document}


% Definition of title page:
\title{Darcs User Manual}
\date{
\darcsVersion
} % icky newline before closing brace is to appease preproc.hs.
\author{David Roundy}

\maketitle

\tableofcontents

\chapter{Introduction}

This manual provides a stable documentation for using darcs.
To find more up-to-date and complementary information, please consult the
darcs wiki at
\htmladdnormallinkfoot{http://wiki.darcs.net}{http://wiki.darcs.net}.

Darcs is a revision control system, along the lines of Subversion, Git or
Mercurial.  That means that it keeps track of various revisions and branches of your
project, allows for changes to propagate from one branch to another.  Darcs
is intended to be an ``advanced'' revision control system.  Darcs has two
particularly distinctive features which differ from other revision control
systems: 1) each copy of the source is a fully functional branch, and 2)
it is based on changes and not revisions.

\paragraph{Every source tree a branch}
The primary simplifying notion of darcs is that \emph{every} copy of your
source code is a full repository. This is dramatically different from Subversion,
in which the normal usage is for there to be one central repository from
which source code will be checked out. It is closer to the notion of arch,
since the `normal' use of arch is for each developer to create his own
repository. However, darcs makes it even easier, since simply checking out
the code is all it takes to create a new repository. This has several
advantages, since you can harness the full power of darcs in any scratch
copy of your code, without committing your possibly destabilizing changes to
a central repository.

%% FIXME: unbound personal pronoun (me).  Use third person, passive voice.

\paragraph{Change-oriented}
The development of a simplified theory of patches is what originally
motivated the creation of darcs. This patch formalism means that darcs patches
have a set of properties, which make possible manipulations that couldn't be
done in other revision control systems. First, every patch is invertible.
Secondly, sequential patches (i.e.\ patches that are created in sequence, one
after the other) can be reordered, although this reordering can fail, which
means the second patch is dependent on the first. Thirdly, patches which are
in parallel (i.e.\ both patches were created by modifying identical trees)
can be merged, and the result of a set of merges is independent of the order
in which the merges are performed. This last property is critical to darcs'
philosophy, as it means that a particular version of a source tree is fully
defined by the list of patches that are in it, i.e.\ there is no issue
regarding the order in which merges are performed. For a more thorough
discussion of darcs' theory of patches, see Appendix~\ref{Patch}.

%% FIXME: clarify - the theory is advanced, the UI is simple.

\paragraph{A simple advanced tool}
Besides being ``advanced'' as discussed above, darcs is actually also quite
simple. Versioning tools can be seen as three layers. At the foundation is
the ability to manipulate changes. On top of that must be placed some kind
of database system to keep track of the changes. Finally, at the very top is
some sort of distribution system for getting changes from one place to
another.

In darcs, only the first of these three layers is of particular interest,
so the other two are done as simply as possible.  At the database
layer, darcs just has an ordered list of patches along with the patches
themselves, each stored as an individual file.  Darcs' distribution system
is based on a dumb server, typically apache or just a local or network file
system when pulling patches.
darcs has built-in support for using \verb!ssh! to write to a remote file
system. A darcs executable is called on the remote system to apply the patches.
Arbitrary other transport protocols are supported, through an environment
variable describing a command that will run darcs on the remote system.
See the documentation for DARCS\_APPLY\_FOO in Chapter~\ref{configuring}
for details.

The recommended method is to send patches through gpg-signed email
messages, which has the advantage of being mostly asynchronous.

\paragraph{Keeping track of changes rather than versions}

In the last paragraph, I explained revision control systems in terms of
three layers.  One can also look at them as having two distinct uses.  One
is to provide a history of previous versions.  The other is to keep track
of changes that are made to the repository, and to allow these changes to
be merged and moved from one repository to another.  These two uses are
distinct, and almost orthogonal, in the sense that a tool can support one
of the two uses optimally while providing no support for the other.  Darcs
is not intended to maintain a history of versions, although it is possible
to kludge together such a revision history, either by making each new patch
depend on all previous patches, or by tagging regularly.  In a sense, this
is what the tag feature is for, but the intention is that tagging will be
used only to mark particularly notable versions (e.g.\ released versions, or
perhaps versions that pass a time consuming test suite).

Other revision control systems are centered upon the job of keeping track
of a history of versions, with the ability to merge changes being added as
it was seen that this would be desirable.  But the fundamental object
remained the versions themselves.

In such a system, a patch (I am using patch here to mean an encapsulated
set of changes) is uniquely determined by two trees.  Merging changes that
are in two trees consists of finding a common parent tree, computing the
diffs of each tree with their parent, and then cleverly combining those two
diffs and applying the combined diff to the parent tree, possibly at some
point in the process allowing human intervention, to allow for fixing up
problems in the merge such as conflicts.

In the world of darcs, the source tree is \emph{not} the fundamental
object, but rather the patch is the fundamental object.  Rather than a
patch being defined in terms of the difference between two trees, a tree is
defined as the result of applying a given set of patches to an empty tree.
Moreover, these patches may be reordered (unless there are dependencies
between the patches involved) without changing the tree.  As a result,
there is no need to find a common parent when performing a merge.  Or, if
you like, their common parent is defined by the set of common patches, and
may not correspond to any version in the version history.

One useful consequence of darcs' patch-oriented philosophy is that since a
patch need not be uniquely defined by a pair of trees (old and new), we can
have several ways of representing the same change, which differ only in how
they commute and what the result of merging them is.  Of course, creating
such a patch will require some sort of user input.  This is a Good Thing,
since the user \emph{creating} the patch should be the one forced to think
about what he really wants to change, rather than the users merging the
patch.  An example of this is the token replace patch (See
Section~\ref{token_replace}).  This feature makes it possible to create a
patch, for example, which changes every instance of the variable
``stupidly\_named\_var'' to ``better\_var\_name'', while leaving
``other\_stupidly\_named\_var'' untouched.  When this patch is merged with
any other patch involving the ``stupidly\_named\_var'', that instance will
also be modified to ``better\_var\_name''.  This is in contrast to a more
conventional merging method which would not only fail to change new
instances of the variable, but would also involve conflicts when merging
with any patch that modifies lines containing the variable.  By more using
additional information about the programmer's intent, darcs is thus able to
make the process of changing a variable name the trivial task that it
really is, which is really just a trivial search and replace, modulo
tokenizing the code appropriately.

The patch formalism discussed in Appendix~\ref{Patch} is what makes darcs'
approach possible.  In order for a tree to consist of a set of patches,
there must be a deterministic merge of any set of patches, regardless of the
order in which they must be merged.  This requires that one be able to
reorder patches.  While I don't know that the patches are required to be
invertible as well, my implementation certainly requires invertibility.  In
particular, invertibility is required to make use of
Theorem~\ref{merge_thm}, which is used extensively in the manipulation of
merges.

\section{Features}

\paragraph{Record changes locally}
In darcs, the equivalent of a svn ``commit'' is called record, because it
doesn't put the change into any remote or centralized repository.  Changes
are always recorded locally, meaning no net access is required in order to
work on your project and record changes as you make them.  Moreover, this
means that there is no need for a separate ``disconnected operation'' mode.

\paragraph{Interactive records}
You can choose to perform an interactive record, in which case darcs will
prompt you for each change you have made and ask if you wish to record it.
Of course, you can tell darcs to record all the changes in a given file, or
to skip all the changes in a given file, or go back to a previous change,
or whatever.  There is also an experimental graphical interface, which
allows you to view and choose changes even more easily, and in whichever
order you like.

\paragraph{Unrecord local changes}
As a corollary to the ``local'' nature of the record operation, if a change
hasn't yet been published to the world---that is, if the local repository
isn't accessible by others---you can safely unrecord a change (even if it
wasn't the most recently recorded change) and then re-record it
differently, for example if you forgot to add a file, introduced a bug or
realized that what you recorded as a single change was really two separate
changes.

\paragraph{Interactive everything else}
Most darcs commands support an interactive interface.  The ``revert''
command, for example, which undoes unrecorded changes has the same
interface as record, so you can easily revert just a single change.  Pull,
push, send and apply all allow you to view and interactively select which
changes you wish to pull, push, send or apply.

\paragraph{Test suites}
Darcs has support for integrating a test suite with a repository.  If you
choose to use this, you can define a test command (e.g.\ ``make check'') and
have darcs run that command on a clean copy of the project either prior to
recording a change or prior to applying changes---and to reject changes
that cause the test to fail.

\paragraph{Any old server}
Darcs does not require a specialized server in order to make a repository
available for read access.  You can use http, ftp, or even just a plain old
ssh server to access your darcs repository.

\paragraph{You decide write permissions}
Darcs doesn't try to manage write access.  That's your business.  Supported
push methods include direct ssh access (if you're willing to \emph{give}
direct ssh access away), using sudo to allow users who already have shell
access to only apply changes to the repository, or verification of
gpg-signed changes sent by email against a list of allowed keys.  In
addition, there is good support for submission of patches by email that
are not automatically applied, but can easily be applied with a shell escape
from a mail reader (this is how I deal with contributions to darcs).

\paragraph{Symmetric repositories}
Every darcs repository is created equal (well, with the exception of a
``partial'' repository, which doesn't contain a full history\ldots), and every
working directory has an associated repository.  As a result, there is a
symmetry between ``uploading'' and ``downloading'' changes---you can use
the same commands (push or pull) for either purpose.

\paragraph{CGI script}
Darcs has a CGI script that allows browsing of the repositories.

\paragraph{Portable}
Darcs runs on UNIX (or UNIX-like) systems (which includes Mac~OS~X) as well
as on Microsoft Windows.

\paragraph{File and directory moves}
Renames or moves of files and directories, of course are handled properly,
so when you rename a file or move it to a different directory, its history
is unbroken, and merges with repositories that don't have the file renamed
will work as expected.

\paragraph{Token replace}
You can use the ``darcs replace'' command to modify all occurrences of a
particular token (defined by a configurable set of characters that are
allowed in ``tokens'') in a file.  This has the advantage that merges with
changes that introduce new copies of the old token will have the effect of
changing it to the new token---which comes in handy when changing a
variable or function name that is used throughout a project.

\paragraph{Configurable defaults}
You can easily configure the default flags passed to any command on either
a per-repository or a per-user basis or a combination thereof.



\chapter{Getting started}

This chapter will lead you through an example use of darcs, which hopefully
will allow you to get started using darcs with your project.

\section{Creating your repository}

Creating your repository in the first place just involves telling darcs to
create the special directory (called {\tt \_darcs}) in your project tree,
which will hold the revision information.  This is done by simply calling
from the root directory of your project:
\begin{verbatim}
$ cd my_project/
$ darcs initialize
\end{verbatim}
This creates the \verb|_darcs| directory and populates it with whatever
files and directories are needed to describe an empty project.  You now
need to tell darcs what files and directories in your project should be
under revision control.  You do this using the command \verb|darcs add|:
\begin{verbatim}
$ darcs add *.c Makefile.am configure.ac
\end{verbatim}
When you have added all your files (or at least, think you have), you will
want to record your changes.  ``Recording'' always includes adding a note
as to why the change was made, or what it does.  In this case, we'll just
note that this is the initial version.
\begin{verbatim}
$ darcs record --all
What is the patch name? Initial revision.
\end{verbatim}
Note that since we didn't specify a patch name on the command line we were
prompted for one.  If the environment variable `EMAIL' isn't set, you will
also be prompted for your email address.  Each patch that is recorded is
given a unique identifier consisting of the patch name, its creator's email
address, and the date when it was created.

\section{Making changes}

Now that we have created our repository, make a change to one or more of
your files.  After making the modification run:
\begin{verbatim}
$ darcs whatsnew
\end{verbatim}
This should show you the modifications that you just made, in the darcs
patch format.  If you prefer to see your changes in a different format,
read Section~\ref{whatsnew}, which describes the whatsnew command in
detail.

Let's say you have now made a change to your project.  The next thing to do
is to record a patch.  Recording a patch consists of grouping together a
set of related changes, and giving them a name.  It also tags the patch
with the date it was recorded and your email address.

To record a patch simply type:
\begin{verbatim}
$ darcs record
\end{verbatim}
darcs will then prompt you with all the changes that you have made that
have not yet been recorded, asking you which ones you want to include in
the new patch.  Finally, darcs will ask you for a name for the patch.

You can now rerun whatsnew, and see that indeed the changes you have
recorded are no longer marked as new.

\section{Making your repository visible to others}
How do you let the world know about these wonderful changes?  Obviously,
they must be able to see your repository.  Currently the easiest way to do
this is typically by http using any web server.  The recommended way to do
this (using apache in a UNIX environment) is to create a directory called
{\tt /var/www/repos}, and then put a symlink to your repository there:
\begin{verbatim}
$ cd /var/www/repos
$ ln -s /home/username/myproject .
\end{verbatim}

\section{Getting changes made to another repository}
Ok, so I can now browse your repository using my web browser\ldots\ so
what? How do I get your changes into \emph{my} repository, where they can
do some good? It couldn't be easier.  I just \verb|cd| into my repository,
and there type:
\begin{verbatim}
$ darcs pull http://your.server.org/repos/yourproject
\end{verbatim}
Darcs will check to see if you have recorded any changes that aren't in my
current repository.  If so, it'll prompt me for each one, to see which ones
I want to add to my repository.  Note that you may see a different series
of prompts depending on your answers, since sometimes one patch depends on
another, so if you answer yes to the first one, you won't be prompted for
the second if the first depends on it.

Of course, maybe I don't even have a copy of your repository.  In that case
I'd want to do a
\begin{verbatim}
$ darcs get http://your.server.org/repos/yourproject
\end{verbatim}
which gets the whole repository.

I could instead create an empty repository and fetch all of your patches
with pull.  Get is just a more efficient way to clone a whole repository.

Get, pull and push also work over ssh.  Ssh-paths are of the same form
accepted by scp, namely \verb|[username@]host:/path/to/repository|.

\section{Moving patches from one repository to another}

Darcs is flexible as to how you move patches from one repository to another.
This section will introduce all the ways you can get patches from one place
to another, starting with the simplest and moving to the most complicated.

\subsection{All pulls}

The simplest method is the ``all-pull'' method.  This involves making each
repository readable (by http, ftp, nfs-mounted disk, whatever), and you
run \verb|darcs pull| in the repository you want to move the patch to.  This is nice,
as it doesn't require you to give write access to anyone else, and is
reasonably simple.

\subsection{Send and apply manually}

Sometimes you have a machine on which it is not convenient to set up a web
server, perhaps because it's behind a firewall or perhaps for security
reasons, or because it is often turned off.  In this case you can use
\verb|darcs send|
from that computer to generate a patch bundle destined for another
repository.  You can either let darcs email the patch for you, or save it
as a file and transfer it by hand.  Then in the destination repository you
(or the owner of that repository) run \verb|darcs apply| to apply the patches contained
in the bundle.  This is also quite a simple method since, like the all-pull
method, it doesn't require that you give anyone write access to your
repository.  But it's less convenient, since you have to keep track of the
patch bundle (in the email, or whatever).

If you use the send and apply method with email, you'll probably want to
create a \verb!_darcs/prefs/email! file containing your email address.
This way anyone who sends to your repository will automatically send the
patch bundle to your email address.

If you receive many patches by email, you probably will benefit by running
darcs apply directly from your mail program.  I have in my \verb!.muttrc!
the following:
\begin{verbatim}
auto_view text/x-patch text/x-darcs-patch
macro pager A "<pipe-entry>darcs apply --verbose --mark-conflicts \
        --reply droundy@abridgegame.org --repodir ~/darcs"
\end{verbatim}
which allows me to view a sent patch, and then apply the patch directly from \verb!mutt!, sending a
confirmation email to the person who sent me the patch. The autoview line relies on on the following
lines, or something like them, being present in one's \verb!.mailcap!:
\begin{verbatim}
text/x-patch;                           cat; copiousoutput
text/x-darcs-patch;                     cat; copiousoutput
\end{verbatim}

\subsection{Push}

If you use ssh (and preferably also ssh-agent, so you won't have to keep
retyping your password), you can use the push method to transfer changes
(using the scp protocol for communication).  This method is again not very
complicated, since you presumably already have the ssh permissions set up.
Push can also be used when the target repository is local, in which case
ssh isn't needed.  On the other hand, in this situation you could as easily
run a pull, so there isn't much benefit.

Note that you can use push to administer a multiple-user repository.  You
just need to create a user for the repository (or repositories), and give
everyone with write access ssh access, perhaps using
\verb!.ssh/authorized_keys!.  Then they run
\begin{verbatim}
$ darcs push repouser@repo.server:repo/directory
\end{verbatim}

\subsection{Push ---apply-as}

Now we get more subtle.  If you like the idea in the previous paragraph
about creating a repository user to own a repository which is writable by
a number of users, you have one other option.

Push \verb!--apply-as! can run on either a local repository or one accessed
with ssh, but uses \verb!sudo! to run a darcs apply command (having created
a patch bundle as in send) as another user.  You can add the following line
in your \verb|sudoers| file to allow the users to apply their patches to a
centralized repository:
{\small
\begin{verbatim}
ALL   ALL = (repo-user) NOPASSWD: /usr/bin/darcs apply --all --repodir /repo/path*
\end{verbatim}
}
This method is ideal for a centralized repository when all the users have
accounts on the same computer, if you don't want your users to be able to
run arbitrary commands as repo-user.

\subsection{Sending signed patches by email}

Most of the previous methods are a bit clumsy if you don't want to give
each person with write access to a repository an account on your server.  Darcs
send can be configured to send a cryptographically signed patch by email.
You can then set up your mail system to have darcs verify that patches were
signed by an authorized user and apply them when a patch is received by
email.  The results of the apply can be returned to the user by email.
Unsigned patches (or patches signed by unauthorized users) will be
forwarded to the repository owner (or whoever you configure them to be
forwarded to\ldots).

This method is especially nice when combined with the \verb!--test! option
of darcs apply, since it allows you to run the test suite (assuming you
have one) and reject patches that fail---and it's all done on the server,
so you can happily go on working on your development machine without
slowdown while the server runs the tests.

Setting up darcs to run automatically in response to email is by far the
most complicated way to get patches from one repository to another\ldots\ so it'll
take a few sections to explain how to go about it.

\paragraph{Security considerations}

When you set up darcs to run apply on signed patches, you should assume
that a user with write access can write to any file or directory that is
writable by the user under which the apply process runs.  Unless you
specify the \verb!--no-test! flag to darcs apply (and this is \emph{not}
the default), you are also allowing anyone with write access to that
repository to run arbitrary code on your machine (since they can run a test
suite---which they can modify however they like).  This is quite a
potential security hole.

For these reasons, if you don't implicitly trust your users, it is
recommended that you create a user for each repository to limit the damage
an attacker can do with access to your repository.  When considering who to
trust, keep in mind that a security breach on any developer's machine could
give an attacker access to their private key and passphrase, and thus to
your repository.

\paragraph{Installing necessary programs}

You also must install the following programs: gnupg, a mailer configured to
receive mail (e.g.\ exim, sendmail or postfix), and a web server (usually
apache).

\paragraph{Granting access to a repository}

You create your gpg key by running (as your normal user):
\begin{verbatim}
$ gpg --gen-key
\end{verbatim}
You will be prompted for your name and email address, among other options.
%%To add your public key to the allowed keys keyring.
Of course, you can
skip this step if you already have a gpg key you wish to use.

You now need to export the public key so we can tell the patcher about it.
You can do this with the following command (again as your normal user):
\begin{verbatim}
$ gpg --export "email@address" > /tmp/exported_key
\end{verbatim}
And now we can add your key to the \verb!allowed_keys!:
\begin{verbatim}
(as root)> gpg --keyring /var/lib/darcs/repos/myproject/allowed_keys \
               --no-default-keyring --import /tmp/exported_key
\end{verbatim}
You can repeat this process any number of times to authorize multiple users
to send patches to the repository.

You should now be able to send a patch to the repository by running as your
normal user, in a working copy of the repository:
\begin{verbatim}
$ darcs send --sign http://your.computer/repos/myproject
\end{verbatim}
You may want to add ``send sign'' to the file \verb!_darcs/prefs/defaults!
so that you won't need to type \verb!--sign! every time you want to
send\ldots

If your gpg key is protected by a passphrase, then executing \verb!send!
with the \verb!--sign! option might give you the following error:
\begin{verbatim}
darcs failed:  Error running external program 'gpg'
\end{verbatim}
The most likely cause of this error is that you have a misconfigured
gpg that tries to automatically use a non-existent gpg-agent
program. GnuPG will still work without gpg-agent when you try to sign
or encrypt your data with a passphrase protected key. However, it will
exit with an error code 2 (\verb!ENOENT!) causing \verb!darcs! to
fail. To fix this, you will need to edit your \verb!~/.gnupg/gpg.conf!
file and comment out or remove the line that says:
\begin{verbatim}
use-agent
\end{verbatim}
If after commenting out or removing the \verb!use-agent! line in your
gpg configuration file you still get the same error, then you probably
have a modified GnuPG with use-agent as a hard-coded option. In that
case, you should change \verb!use-agent! to \verb!no-use-agent! to
disable it explicitly.

\paragraph{Setting up a sendable repository using procmail}
If you don't have root access on your machine, or perhaps simply don't want
to bother creating a separate user, you can set up a darcs repository using
procmail to filter your mail.  I will assume that you already use procmail
to filter your email.  If not, you will need to read up on it, or perhaps
should use a different method for routing the email to darcs.

To begin with, you must configure your repository so that a darcs send to
your repository will know where to send the email.  Do this by creating a
file in \verb!/path/to/your/repo/_darcs/prefs! called \verb!email!
containing your email address.  As a trick (to be explained below), we will
create the email address with ``darcs repo'' as your name, in an email
address of the form ``David Roundy $<$droundy@abridgegame.org$>$.''
\begin{verbatim}
$ echo 'my darcs repo <user@host.com>' \
      > /path/to/your/repo/_darcs/prefs/email
\end{verbatim}

The next step is to set up a gnupg keyring containing the public keys of
people authorized to send to your repository.  Here I'll give a second way of
going about this (see above for the first).  This time I'll assume you
want to give me write access to your repository.  You can do this by:
\begin{verbatim}
gpg --no-default-keyring \
    --keyring /path/to/the/allowed_keys --recv-keys D3D5BCEC
\end{verbatim}
This works because ``D3D5BCEC'' is the ID of my gpg key, and I have
uploaded my key to the gpg keyservers.  Actually, this also requires that
you have configured gpg to access a valid keyserver.  You can, of course,
repeat this command for all keys you want to allow access to.

Finally, we add a few lines to your \verb!.procmailrc!:
\begin{verbatim}
:0
* ^TOmy darcs repo
|(umask 022; darcs apply --reply user@host.com \
    --repodir /path/to/your/repo --verify /path/to/the/allowed_keys)
\end{verbatim}
The purpose for the ``my darcs repo'' trick is partially to make it easier
to recognize patches sent to the repository, but is even more crucial to
avoid nasty bounce loops by making the \verb!--reply! option have an email
address that won't go back to the repository.  This means that unsigned
patches that are sent to your repository will be forwarded to your ordinary
email.

Like most mail-processing programs, Procmail by default sets a tight umask.
However, this will prevent the repository from remaining world-readable;
thus, the ``umask 022'' is required to relax the umask.
(Alternatively, you could set Procmail's global \verb!UMASK! variable
to a more suitable value.)

\paragraph{Checking if your e-mail patch was applied}

After sending a patch with \verb!darcs send!, you may not receive any feedback,
even if the patch is applied. You can confirm whether or not your patch was applied
to the remote repository by pointing \verb!darcs changes! at a remote repository:
\begin{verbatim}
darcs changes --last=10 --repo=http://darcs.net/
\end{verbatim}

That shows you the last 10 changes in the remote repository. You can adjust the options given
to \verb!changes! if a more advanced query is needed.

%  Copyright (C) 2004 David Roundy
%
%  This program is free software; you can redistribute it and/or modify
%  it under the terms of the GNU General Public License as published by
%  the Free Software Foundation; either version 2, or (at your option)
%  any later version.
%
%  This program is distributed in the hope that it will be useful,
%  but WITHOUT ANY WARRANTY; without even the implied warranty of
%  MERCHANTABILITY or FITNESS FOR A PARTICULAR PURPOSE.  See the
%  GNU General Public License for more details.
%
%  You should have received a copy of the GNU General Public License
%  along with this program; if not, write to the Free Software Foundation,
%  Inc., 59 Temple Place - Suite 330, Boston, MA 02111-1307, USA.

\chapter{Configuring darcs}\label{configuring}

There are several ways you can adjust darcs' behavior to suit your needs.
The first is to edit files in the \verb!_darcs/prefs/! directory of a
repository.  Such configuration only applies when working with that
repository.  To configure darcs on a per-user rather than per-repository
basis (but with essentially the same methods), you can edit (or create)
files in the \verb!~/.darcs/! directory.
Finally, the behavior of some darcs commands can be modified by setting
appropriate environment variables.

\paragraph{Microsoft Windows}\label{ms_win}

The global darcs directory is \verb!%APPDATA%\darcs\!.  This typically expands to
\texttt{C:\textbackslash{}Documents And Settings\textbackslash{}\emph{user}\textbackslash{}Application Data\textbackslash{}darcs\textbackslash{}}.
This folder contains the cache, as well as all the per-user
settings files: preferences, boring etc... These will became the new defaults
that can be overridden on per-repository basis.

\input{Darcs/Repository/Prefs.lhs}

\input{Darcs/Repository/Motd.lhs}

\section{Environment variables}

There are a few environment variables whose contents affect darcs'
behavior.  Here is a quick list of all the variables and their
documentation in the rest of the manual:

\begin{tabular}{|l|r|}
\hline
\textbf{Variable} & \textbf{Section} \\
\hline
DARCS\_EDITOR, EDITOR, VISUAL & \ref{env:DARCS_EDITOR} \\
DARCS\_PAGER, PAGER &  \ref{env:DARCS_PAGER} \\
HOME & \ref{env:HOME} \\
TERM & \ref{env:TERM} \\
\hline
DARCS\_EMAIL, EMAIL  & \ref{env:DARCS_EMAIL} \\
\hline
DARCS\_APPLY\_FOO & \ref{env:DARCS_X_FOO} \\
DARCS\_GET\_FOO & \ref{env:DARCS_X_FOO} \\
DARCS\_MGET\_FOO & \ref{env:DARCS_X_FOO} \\
DARCS\_MGETMAX & \ref{env:DARCS_MGETMAX} \\
DARCS\_PROXYUSERPWD & \ref{env:DARCS_PROXYUSERPWD} \\
DARCS\_CONNECTION\_TIMEOUT & \ref{env:DARCS_CONNECTION_TIMEOUT}\\
DARCS\_SSH & \ref{env:DARCS_SSH} \\
DARCS\_SCP & \ref{env:DARCS_SCP} \\
DARCS\_SFTP & \ref{env:DARCS_SFTP} \\
SSH\_PORT & \ref{env:SSH_PORT} \\
\hline
DARCS\_ALTERNATIVE\_COLOR & \ref{env:DARCS_ALWAYS_COLOR}\\
DARCS\_ALWAYS\_COLOR & \ref{env:DARCS_ALWAYS_COLOR}\\
DARCS\_DO\_COLOR\_LINES & \ref{env:DARCS_DO_COLOR_LINES}\\
DARCS\_DONT\_COLOR   & \ref{env:DARCS_ALWAYS_COLOR} \\
DARCS\_DONT\_ESCAPE\_TRAILING\_CR     & \ref{env:DARCS_DONT_ESCAPE_white}\\
DARCS\_DONT\_ESCAPE\_TRAILING\_SPACES & \ref{env:DARCS_DONT_ESCAPE_white} \\
DARCS\_DONT\_ESCAPE\_8BIT & \ref{env:DARCS_DONT_ESCAPE_nonascii}\\
DARCS\_DONT\_ESCAPE\_ANYTHING & \ref{env:DARCS_DONT_ESCAPE_nonascii}\\
DARCS\_DONT\_ESCAPE\_ISPRINT & \ref{env:DARCS_DONT_ESCAPE_nonascii}\\
DARCS\_ESCAPE\_EXTRA & \ref{env:DARCS_DONT_ESCAPE_nonascii}\\
DARCS\_DONT\_ESCAPE\_EXTRA & \ref{env:DARCS_DONT_ESCAPE_nonascii}\\
\hline
\end{tabular}

\section{General-purpose variables}

\darcsEnv{DARCS_EDITOR}
\darcsEnv{DARCS_PAGER}
\darcsEnv{DARCS_TMPDIR}
\darcsEnv{DARCS_KEEP_TMPDIR}
\darcsEnv{HOME}

\section{Remote repositories}
\paragraph{DARCS\_CONNECTION\_TIMEOUT}
\label{env:DARCS_CONNECTION_TIMEOUT}
Set the maximum time in seconds that darcs allows and connection to
take. If the variable is not specified the default are 30 seconds. This
option only works with curl.
\darcsEnv{DARCS_SSH}
\darcsEnv{DARCS_SCP}
\darcsEnv{SSH_PORT}
\darcsEnv{HTTP_PROXY}
\darcsEnv{DARCS_PROXYUSERPWD}

\paragraph{DARCS\_GET\_FOO, DARCS\_MGET\_FOO and DARCS\_APPLY\_FOO}
\label{env:DARCS_X_FOO}
When trying to access a repository with a URL beginning foo://,
darcs will invoke the program specified by the DARCS\_GET\_FOO
environment variable (if defined) to download each file, and the
command specified by the DARCS\_APPLY\_FOO environment variable (if
defined) when pushing to a foo:// URL.  

This method overrides all other ways of getting \verb!foo://xxx! URLs.

Note that each command should be constructed so that it sends the downloaded
content to STDOUT, and the next argument to it should be the URL\@.  Here are some
examples that should work for DARCS\_GET\_HTTP:

\begin{verbatim}
fetch -q -o -  
curl -s -f
lynx -source 
wget -q -O -
\end{verbatim}

Apart from such toy examples, it is likely that you will need to
manipulate the argument before passing it to the actual fetcher
program.  For example, consider the problem of getting read access to
a repository on a CIFS (SMB) share without mount privileges:

\begin{verbatim}
export DARCS_GET_SMB="smbclient -c get"
darcs get smb://fs/twb/Desktop/hello-world
\end{verbatim}

The above command will not work for several reasons.  Firstly, Darcs
will pass it an argument beginning with `smb:', which smbclient does
not understand.  Secondly, the host and share `//fs/twb' must be
presented as a separate argument to the path `Desktop/hello-world'.
Thirdly, smbclient requires that `get' and the path be a single
argument (including a space), rather than two separate arguments.
Finally, smbclient's `get' command writes the file to disk, while
Darcs expects it to be printed to standard output.

In principle, we could get around such problems by making the variable
contain a shell script, e.g.

\begin{verbatim}
export DARCS_GET_SMB='sh -c "...; smbclient $x -c \"get $y\""'
\end{verbatim}

Unfortunately, Darcs splits the command on whitespace and does not
understand that quotation or escaping, so there is no way to make
Darcs pass the text after `-c' to sh as a single argument.  Therefore,
we instead need to put such one-liners in separate, executable scripts.

Continuing our smbclient example, we create an executable script
\verb|~/.darcs/libexec/get_smb| with the following contents:

\begin{verbatim}
#!/bin/bash -e
IFS=/ read host share file <<<"${1#smb://}"
smbclient //$host/$share -c "get $file -"
\end{verbatim}

And at last we can say

\begin{verbatim}
export DARCS_GET_SMB=~/.darcs/libexec/get_smb
darcs get smb://fs/twb/Desktop/hello-world
\end{verbatim}


If set, DARCS\_MGET\_FOO
will be used to fetch many files from a single repository simultaneously.
Replace FOO and foo as appropriate to handle other URL schemes.
These commands are \emph{not} interpreted by a shell, so you cannot
use shell metacharacters, and the first word in the command must
be the name of an executable located in your path. The GET command
will be called with a URL for each file.  The MGET command will be
invoked with a number of URLs and is expected to download the files
to the current directory, preserving the file name but not the path.
The APPLY command will be called with a darcs patchfile piped into
its standard input. Example:

\begin{verbatim}
wget -q 
\end{verbatim}

\paragraph{DARCS\_MGETMAX}
\label{env:DARCS_MGETMAX}
When invoking a DARCS\_MGET\_FOO command, darcs will limit the
number of URLs presented to the command to the value of this variable,
if set, or 200.

These commands are \emph{not} interpreted by a shell, so you cannot use shell
meta-characters.

\section{Highlighted output}
\label{env:DARCS_ALWAYS_COLOR}
\label{env:DARCS_DO_COLOR_LINES}
\label{env:DARCS_DONT_ESCAPE_white}

If the terminal understands ANSI color escape sequences,
darcs will highlight certain keywords and delimiters when printing patches.
This can be turned off by setting the environment variable DARCS\_DONT\_COLOR to 1.
If you use a pager that happens to understand ANSI colors, like \verb!less -R!,
darcs can be forced always to highlight the output
by setting DARCS\_ALWAYS\_COLOR to 1.
If you can't see colors you can set DARCS\_ALTERNATIVE\_COLOR to 1,
and darcs will use ANSI codes for bold and reverse video instead of colors.
In addition, there is an extra-colorful mode, which is not enabled by
default, which can be activated with DARCS\_DO\_COLOR\_LINES.

By default darcs will escape (by highlighting if possible) any kind of spaces at the end of lines
when showing patch contents.
If you don't want this you can turn it off by setting
DARCS\_DONT\_ESCAPE\_TRAILING\_SPACES to 1.
A special case exists for only carriage returns:
DARCS\_DONT\_ESCAPE\_TRAILING\_CR.


\section{Character escaping and non-ASCII character encodings}
\label{env:DARCS_DONT_ESCAPE_nonascii}

Darcs needs to escape certain characters when printing patch contents to a terminal.
Characters like \emph{backspace} can otherwise hide patch content from the user,
and other character sequences can even in some cases redirect commands to the shell
if the terminal allows it.

By default darcs will only allow printable 7-bit ASCII characters (including space),
and the two control characters \emph{tab} and \emph{newline}.
(See the last paragraph in this section for a way to tailor this behavior.)
All other octets are printed in quoted form (as \verb!^<control letter>! or
\verb!\!\verb!<hex code>!).

Darcs has some limited support for locales.
If the system's locale is a single-byte character encoding,
like the Latin encodings,
you can set the environment variable DARCS\_DONT\_ESCAPE\_ISPRINT to 1
and darcs will display all the printables in the current system locale
instead of just the ASCII ones.
NOTE: This curently does not work on some architectures if darcs is
compiled with GHC~6.4 or later. Some non-ASCII control characters might be printed
and can possibly spoof the terminal.

For multi-byte character encodings things are less smooth.
UTF-8 will work if you set DARCS\_DONT\_ESCAPE\_8BIT to 1,
but non-printables outside the 7-bit ASCII range are no longer escaped.
E.g., the extra control characters from Latin-1
might leave your terminal at the mercy of the patch contents.
Space characters outside the 7-bit ASCII range are no longer recognized
and will not be properly escaped at line endings.

As a last resort you can set DARCS\_DONT\_ESCAPE\_ANYTHING to 1.
Then everything that doesn't flip code sets should work,
and so will all the bells and whistles in your terminal.
This environment variable can also be handy
if you pipe the output to a pager or external filter
that knows better than darcs how to handle your encoding.
Note that \emph{all} escaping,
including the special escaping of any line ending spaces,
will be turned off by this setting.

There are two environment variables you can set
to explicitly tell darcs to not escape or escape octets.
They are
DARCS\_DONT\_ESCAPE\_EXTRA and DARCS\_ESCAPE\_EXTRA.
Their values should be strings consisting of the verbatim octets in question.
The do-escapes take precedence over the dont-escapes.
Space characters are still escaped at line endings though.
The special environment variable DARCS\_DONT\_ESCAPE\_TRAILING\_CR
turns off escaping of carriage return last on the line (DOS style).


% This file (unlike the rest of darcs) is in the public domain.
 

\chapter{Best practices}

This chapter is intended to review various scenarios and describe in each
case effective ways of using darcs.  There is no one ``best practice'', and
darcs is a sufficiently low-level tool that there are many high-level ways
one can use it, which can be confusing to new users.  The plan (and hope)
is that various users will contribute here describing how they use darcs in
different environments.  However, this is not a wiki, and contributions
will be edited and reviewed for consistency and wisdom.


\section{Creating patches}

This section will lay down the concepts around patch creation.
The aim is to develop a way of thinking
that corresponds well to how darcs is behaving
--- even in complicated situations.

	In a single darcs repository you can think of two ``versions'' of the source tree.
	They are called the \emph{working} and \emph{pristine} trees.
    \emph{Working} is your normal source tree, with or without darcs alongside.
	The only thing that makes it part of a darcs repository
	is the \verb!_darcs! directory in its root.
    \emph{Pristine} is the recorded state of the source tree.
	The pristine tree is constructed from groups of changes,
        called {\em patches\/} (some other version control systems use the
	term {\em changeset\/} instead of {\em patch\/}).\footnote{If
	you look inside \_darcs you will find files or directories named
	{\tt patches} and {\tt inventories}, which store all the patches
          ever recorded.  If the repository holds a cached pristine tree, it
          is stored in a directory called {\tt pristine} or {\tt current\/};
          otherwise, the fact that there is no pristine tree is marked
          by the presence of a file called {\tt pristine.none} or {\tt
            current.none}.}
	Darcs will create and store these patches
	based on the changes you make in \emph{working}.


\subsection{Changes}
	If \emph{working} and \emph{pristine} are the same,
	there are ``no changes'' in the repository.
	Changes can be introduced (or removed) by editing the files in \emph{working}.
	They can also be caused by darcs commands,
	which can modify \emph{both} \emph{working} and \emph{pristine}.
	It is important to understand for each darcs command
	how it modifies \emph{working}, \emph{pristine} or both of them.

	\verb!whatsnew! (as well as \verb!diff!) can show
	the difference between \emph{working} and \emph{pristine} to you.
	It will be shown as a difference in \emph{working}.
	In advanced cases it need \emph{not} be \emph{working} that has changed;
	it can just as well have been \emph{pristine}, or both.
	The important thing is the difference and what darcs can do with it.

\subsection{Keeping or discarding changes}
    If you have a difference in \emph{working}, you do two things
    with it: \verb!record! it to keep it, or \verb!revert! it to lose the changes.%
		\footnote{%
		Revert can undo precious work in a blink.
		To protect you from great grief,
		the discarded changes are saved temporarily
		so the latest revert can be undone with unrevert.}

	If you have a difference between \emph{working} and \emph{pristine}%
	---for example after editing some files in \emph{working}---%
	\verb!whatsnew! will show some ``unrecorded changes''.
	To save these changes, use \verb!record!.
	It will create a new patch in \emph{pristine} with the same changes,
	so \emph{working} and \emph{pristine} are no longer different.
	To instead undo the changes in \emph{working}, use \verb!revert!.
	It will modify the files in \emph{working} to be the same as in \emph{pristine}
	(where the changes do not exist).


\subsection{Unrecording changes}
    \verb!unrecord! is a command meant to be run only in private
    repositories. Its intended purpose is to allow developers the flexibility
    to undo patches that haven't been distributed yet.

    However, darcs does not prevent you from unrecording a patch that
    has been copied to another repository. Be aware of this danger!

	If you \verb!unrecord! a patch, that patch will be deleted from \emph{pristine}.
	This will cause \emph{working} to be different from \emph{pristine},
	and \verb!whatsnew! to report unrecorded changes.
	The difference will be the same as just before that patch was \verb!record!ed.
	Think about it.
	\verb!record! examines what's different with \emph{working}
	and constructs a patch with the same changes in \emph{pristine}
	so they are no longer different.
	\verb!unrecord! deletes this patch;
	the changes in \emph{pristine} disappear and the difference is back.

	If the recorded changes included an error,
	the resulting flawed patch can be unrecorded.
	When the changes have been fixed,
	they can be recorded again as a new---hopefully flawless---patch.

	If the whole change was wrong it can be discarded from \emph{working} too,
	with \verb!revert!.
	\verb!revert! will update \emph{working} to the state of \emph{pristine},
	in which the changes do no longer exist after the patch was deleted.

	Keep in mind that the patches are your history,
	so deleting them with \verb!unrecord! makes it impossible to track
	what changes you \emph{really} made.
	Redoing the patches is how you ``cover the tracks''.
	On the other hand,
	it can be a very convenient way to manage and organize changes
	while you try them out in your private repository.
	When all is ready for shipping,
	the changes can be reorganized in what seems as useful and impressive patches.
	Use it with care.

	All patches are global,
	so don't \emph{ever} replace an already ``shipped'' patch in this way!
	If an erroneous patch is deleted and replaced with a better one,
	you have to replace it in \emph{all} repositories that have a copy of it.
	This may not be feasible, unless it's all private repositories.
	If other developers have already made patches or tags in their repositories
	that depend on the old patch, things will get complicated.


\subsection{Special patches and pending}

The patches described in the previous sections have mostly been hunks.
A \emph{hunk} is one of darcs' primitive patch types,
and it is used to remove old lines and/or insert new lines.
There are other types of primitive patches,
such as \emph{adddir} and \emph{addfile}
which add new directories and files,
and \emph{replace}
which does a search-and-replace on tokens in files.

Hunks are always calculated in place with a diff algorithm
just before \verb!whatsnew! or \verb!record!.
But other types of primitive patches need to be explicitly created
with a darcs command.
They are kept in \emph{pending}%
\footnote{In the file {\tt\_darcs/patches/pending}.}
until they are either recorded or reverted.

\emph{Pending} can be thought of as a special extension of \emph{working}.
When you issue, e.g., a darcs \verb!replace! command,
the replace is performed on the files in \emph{working}
and at the same time a replace patch is put in \emph{pending}.
Patches in \emph{pending} describe special changes made in \emph{working}.
The diff algorithm will fictively apply these changes to \emph{pristine}
before it compares it to \emph{working},
so all lines in \emph{working} that are changed by a \verb!replace! command
will also be changed in \emph{pending}$+$\emph{pristine}
when the hunks are calculated.
That's why no hunks with the replaced lines will be shown by \verb!whatsnew!;
it only shows the replace patch in \emph{pending} responsible for the change.

If a special patch is recorded, it will simply be moved to \emph{pristine}.
If it is instead reverted, it will be deleted from \emph{pending}
and the accompanying change will be removed from \emph{working}.

Note that reverting a patch in pending is \emph{not} the same as
simply removing it from pending.
It actually applies the inverse of the change to \emph{working}.
Most notable is that reverting an addfile patch
will delete the file in \emph{working} (the inverse of adding it).
So if you add the wrong file to darcs by mistake,
\emph{don't} \verb!revert! the addfile.
Instead use \verb!remove!, which cancels out the addfile in pending.


\section{Using patches} % still basics

This section will lay down the concepts around patch distribution and branches.
The aim is to develop a way of thinking
that corresponds well to how darcs is behaving
--- even in complicated situations.

A repository is a collection of patches.
Patches have no defined order,
but patches can have dependencies on other patches.
Patches can be added to a repository in any order
as long as all patches depended upon are there.
Patches can be removed from a repository in any order,
as long as no remaining patches depend on them.

Repositories can be cloned to create branches.
Patches created in different branches may conflict.
A conflict is a valid state of a repository.
A conflict makes the working tree ambiguous until the conflict is resolved.


\subsection{Dependencies}

There are two kinds of dependencies:
implicit dependencies and explicit dependencies.

Implicit dependencies is the far most common kind.
These are calculated automatically by darcs.
If a patch removes a file or a line of code,
it will have to depend on the patch that added that file or line of code.\footnote{%
Actually it doesn't have to---in theory---,
but in practice it's hard to create ``negative'' files or lines in the working tree.
See the chapter about Theory of patches for other constraints.}
If a patch adds a line of code,
it will usually have to depend on the patch or patches that added the adjacent lines.

Explicit dependencies can be created if you give the \verb|--ask-deps| option to \verb|darcs record|.
This is good for assuring that logical dependencies hold between patches.
It can also be used to group patches---%
a patch with explicit dependencies doesn't need to change anything---%
and pulling the patch also pulls all patches it was made to depend on.


\subsection{Branches: just normal repositories}

Darcs does not have branches---it doesn't need to.
Every repository can be used as a branch.
This means that any two repositories are ``branches'' in darcs,
but it is not of much use unless they have a large portion of patches in common.
If they are different projects they will have nothing in common,
but darcs may still very well be able to merge them,
although the result probably is nonsense.
Therefore the word ``branch'' isn't a technical term in darcs;
it's just the way we think of one repository in relation to another.

Branches are \emph{very} useful in darcs.
They are in fact \emph{necessary} if you want to do more than only simple work.
When you \verb|get| someone's repository from the Internet,
you are actually creating a branch of it.
But darcs is designed this way, and it has means to make it efficient.
The answer to many questions about how to do a thing with darcs is: ``use a branch''.
It is a simple and elegant solution with great power and flexibility,
which contributes to darcs' uncomplicated user interface.

You create new branches (i.e., clone repositories)
with the \verb|get| and \verb|put| commands.


\subsection{Moving patches around---no versions}

Patches are global, and a copy of a patch either is or is not present in a branch.
This way you can rig a branch almost any way you like,
as long as dependencies are fulfilled---%
darcs \emph{won't} let you break dependencies.
If you suspect a certain feature from some time ago introduced a bug,
you can remove the patch/patches that adds the feature,
and try without it.\footnote{%
darcs even has a special command, {\tt trackdown}
that automatically removes patches
until a specified test no longer fails.}

Patches are added to a repository with \verb|pull|
and removed from the repositories with \verb|obliterate|.
Don't confuse these two commands with \verb|record| and \verb|unrecord|,
which constructs and deconstructs patches.

It is important not to lose patches when (re)moving them around.
\verb|pull| needs a source repository to copy the patch from,
whereas \verb|obliterate| just erases the patch.
Beware that if you obliterate \emph{all} copies of a patch
it is completely lost---forever.
Therefore you should work with branches when you obliterate patches.
The \verb|obliterate| command can wisely be disabled in a dedicated main repository
by adding \verb|obliterate disable| to the repository's defaults file.

For convenience, there is a \verb|push| command.
It works like \verb|pull| but in the other direction.
It also differs from \verb|pull| in an important way:
it starts a second instance of darcs to apply the patch in the target repository,
even if it's on the same computer.
It can cause surprises if you have a ``wrong'' darcs in your PATH.


\subsection{Tags---versions}

While \verb|pull| and \verb|obliterate| can be used to
construct different ``versions'' in a repository,
it is often desirable to name specific configurations of patches
so they can be identified and retrieved easily later.
This is how darcs implements what is usually known as versions.
The command for this is \verb|tag|,
and it records a tag in the current repository.

A tag is just a patch, but it only contains explicit dependencies.
It will depend on all the patches in the current repository.\footnote{%
It will omit patches already depended upon by other patches,
since they will be indirectly depended upon anyway.}
Darcs can recognize if a patch is as a tag;
tags are sometimes treated specially by darcs commands.

While traditional revision control systems tag versions in the time line history,
darcs lets you tag any configuration of patches at any time,
and pass the tags around between branches.

With the option \verb|--tag| to \verb|get| you can easily get
a named version in the repository
as a new branch.


\subsection{Conflicts}

This part of darcs becomes a bit complicated,
and the description given here is slightly simplified.

Conflicting patches are created when
you record changes to the same line in two different repositories.
Same line does \emph{not} mean the same line number and file name,
but the same line added by a common depended-upon patch.

If you are using a darcs-2 repository (Section \ref{initialize}),
darcs does \emph{not} consider two patches making the \emph{same} change to be a
conflict, much in the same fashion as other version control systems.
(The caveat here is two non-identical patches with some identical
changes may conflict.  For the most part, darcs should just do what you
expect).

A conflict \emph{happens} when two conflicting patches meet in the same repository.
This is no problem for darcs; it can happily pull together just any patches.
But it is a problem for the files in \emph{working} (and \emph{pristine}).
The conflict can be thought of as
two patches telling darcs different things about what a file should look like.

Darcs escapes this problem
by ignoring those parts\footnote{%
The primitive patches making up the total patch.}
of the patches that conflict.
They are ignored in \emph{both} patches.
If patch~A changes the line ``FIXME'' to ``FIXED'',
and patch~B changes the same line to ``DONE'',
the two patches together will produce the line ``FIXME''.
Darcs doesn't care which one you pulled into the repository first,
you still get the same result when the conflicting patches meet.
All other changes made by A and B are performed as normal.

Darcs can mark a conflict for you in \emph{working}.
This is done with \verb|mark-conflicts|.
Conflicts are marked such that both conflicting changes
are inserted with special delimiter lines around them.
Then you can merge the two changes by hand,
and remove the delimiters.

When you pull patches,
darcs automatically performs a \verb|mark-conflicts| for you if a conflict happens.
You can remove the markup with \verb|revert|,
Remember that the result will be the lines from
the previous version common to both conflicting patches.
The conflict marking can be redone again with \verb|mark-conflicts|.

A special case is when a pulled patch conflicts with unrecorded changes in the repository.
The conflict will be automatically marked as usual,
but since the markup is \emph{also} an unrecorded change,
it will get mixed in with your unrecorded changes.
There is no guarantee you can revert \emph{only} the markup after this,
and \verb|resolve| will not be able to redo this markup later if you remove it.
It is good practice to record important changes before pulling.

\verb|mark-conflicts| can't mark complicated conflicts.
In that case you'll have to use \verb|darcs diff| and other commands
to understand what the conflict is all about.
If for example two conflicting patches create the same file,
\verb|mark-conflicts| will pick just one of them,
and no delimiters are inserted.
So watch out if darcs tells you about a conflict.

\verb|mark-conflicts| can also be used to check for unresolved conflicts.
If there are none, darcs replies ``No conflicts to resolve''.
While \verb|pull| reports when a conflict happens,
\verb|obliterate| and \verb|get| don't.


\subsection{Resolving conflicts}

A conflict is resolved
(not marked, as with the command \verb|mark-conflicts|)
as soon as some new patch depends on the conflicting patches.
This will usually be the resolve patch you record after manually putting together the pieces
from the conflict markup produced by \verb|mark-conflicts| (or \verb|pull|).
But it can just as well be a tag.
So don't forget to fix conflicts before you accidently ``resolve'' them by recording other patches.

If the conflict is with one of your not-yet-published patches,
you may choose to amend that patch rather than creating a resolve patch.

If you want to back out and wait with the conflict,
you can \verb|obliterate| the conflicting patch you just pulled.
Before you can do that you have to \verb|revert| the conflict markups
that \verb|pull| inserted when the conflict happened.

\section{Use a Global Cache}

When working with darcs 2 it is recommended to use a global cache, as this
is one of the biggest performance enhancing tools of darcs 2.  The global
cache acts as a giant patch pool where darcs first looks for a patch when
grabbing new patches. This saves time by not downloading the same patch
twice from a remote server. It also saves space by storing the patch only
once, if you ensure your cache and your repositories are on the same
hardlink-supporting filesystem. 

Darcs now enables a global patch cache under your home directory by default.
Older darcs 2.x versions required this manual step:

\begin{verbatim}
$ mkdir -p $HOME/.darcs/cache
$ echo cache:$HOME/.darcs/cache > $HOME/.darcs/sources
\end{verbatim}

On MS Windows~\ref{ms_win}, using \verb|cmd.exe| (Command Prompt under Accessories):

\begin{verbatim}
> md "%UserProfile%\Application Data\darcs\cache" (notice double quotes!)
> echo cache:%UserProfile%\Application Data\darcs\cache > "%UserProfile%\Application Data\darcs\sources"
\end{verbatim}

There are some other advanced things you can do in \verb!_darcs/prefs/sources!,
such as create per-repository caches, read-only caches and even set a
primary source repository above any used in a \verb|darcs get| or 
\verb|darcs pull| command.

\subsection{Per-repository caches}
Each time a repository is get, its location is added as an
entry in \_darcs/prefs/sources.  If one of these repositories were to
become totally or temporarily unreachable, it can cause darcs to hang
for a long time trying to reach it.  Fortunately darcs has a mechanism
which helps us to deal with that problem: if an unreachable entry is
discovered, darcs stops using it for the rest of the session and then
notifies to the user to take further action. It will display a message
like the following.\\
\begin{verbatim}
> I could not reach the following repository:
> http://darcs.net/
> If you're not using it, you should probably delete
> the corresponding entry from _darcs/prefs/sources.
\end{verbatim}

\section{Distributed development with one primary developer}
\label{darcs-development-practices}

This is how darcs itself is developed.  There are many contributors to
darcs, but every contribution is reviewed and manually applied by myself.
For this sort of a situation, \verb|darcs send| is ideal, since the barrier for
contributions is very low, which helps encourage contributors.

One could simply set the \verb!_darcs/prefs/email! value to the project
mailing list, but I also use darcs send to send my changes to the main
server, so instead the email address is set to
``\verb!Davids Darcs Repo <droundy@abridgegame.org>!''.  My
\verb-.procmailrc-
file on the server has the following rule:
\begin{verbatim}
:0
* ^TODavids Darcs Repo
|(umask 022; darcs apply --reply darcs-devel@abridgegame.org \
             --repodir /path/to/repo --verify /path/to/allowed_keys)
\end{verbatim}
This causes darcs apply to be run on any email sent to ``Davids Darcs
Repo''.
\verb'apply' actually applies them only if they are signed by an
authorized key.  Currently, the only authorized key is mine, but of course
this could be extended easily enough.

The central darcs repository contains the following values in its
\verb!_darcs/prefs/defaults!:
\begin{verbatim}
apply test
apply verbose
apply happy-forwarding
\end{verbatim}
The first line tells apply to always run the test suite.  The test suite is
in fact the main reason I use send rather than push, since it allows me to
easily continue working (or put my computer to sleep) while the tests are
being run on the main server.  The second line is just there to improve the
email response that I get when a patch has either been applied or failed
the tests.  The third line makes darcs not complain about unsigned patches,
but just to forward them to \verb!darcs-devel!.

On my development computer, I have in my \verb!.muttrc! the following
alias, which allows me to easily apply patches that I get via email
directly to my darcs working directory:
\begin{verbatim}
macro pager A "<pipe-entry>(umask 022; darcs apply --no-test -v \
        --repodir ~/darcs)"
\end{verbatim}


\section{Development by a small group of developers in one office}
\label{dft-development-practices}

This section describes the development method used for the density
functional theory code DFT++, which is available at
\verb!http://dft.physics.cornell.edu/dft!.

We have a number of workstations which all mount the same \verb!/home! via NFS.
We created a special ``dft'' user, with the central repository living in that
user's home directory.  The ssh public keys of authorized persons are added to
the ``dft'' user's \verb!.ssh/allowed_keys!, and we commit patches to this
repository using
\verb'darcs push'.  As in Section~\ref{darcs-development-practices},
we have the central repository set to run the test suite before the push goes
through.

Note that no one ever runs as the dft user.

A subtlety that we ran into showed up in the running of the test suite.
Since our test suite includes the running of MPI programs, it must be run
in a directory that is mounted across our cluster.  To achieve this, we set
the \verb!$DARCS_TMPDIR! % following is added to make emacs color right:$
environment variable to \verb!~/tmp!.

Note that even though there are only four active developers at the moment,
the distributed nature of darcs still plays a large role.  Each developer
works on a feature until it is stable, a process that often takes quite a
few patches, and only once it is stable does he
\verb'push' to the central repository.

\section{Personal development}

It's easy to have several personal development trees using darcs, even
when working on a team or with shared code.  The number and method of
using each personal tree is limited only by such grand limitations as:
your disk space, your imagination, available time, etc.

For example, if the central darcs repository for your development team
is $R_{c}$, you can create a local working directory for feature
$f_1$.  This local working directory contains a full copy of $R_c$
(as of the time of the ``darcs get'' operation) and can be denoted
$R_1$.  In the midst of working on feature $f_1$, you realize it
requires the development of a separate feature $f_2$.  Rather than
intermingling $f_1$ and $f_2$ in the same working tree, you can create
a new working tree for working on $f_2$, where that working tree
contains the repository $R_2$.

While working on $f_2$, other developers may have made other changes;
these changes can be retrieved on a per-patch selection basis by
periodic ``darcs pull'' operations.

When your work on $f_2$ is completed, you can publish it for the use
of other developers by a ``darcs push'' (or ``darcs send'') from $R_2$
to $R_c$.  Independently of the publishing of $f_2$, you can merge
your $f_2$ work into your $f_1$ working tree by a ``darcs pull $R_2$''
in the $R_1$ development tree (or ``darcs push'' from $R_2$ to $R_1$).

When your work on $f_1$ is completed, you publish it as well by a
``darcs push'' from $R_1$ to $R_c$.  

Your local feature development efforts for $f_1$ or $f_2$ can each
consist of multiple patches.  When pushing or pulling to other trees,
these patches can either all be selected or specific patches can be
selected.  Thus, if you introduce a set of debugging calls into the
code, you can commit the debugging code in a distictly separate patch
(or patches) that you will not push to $R_c$.

\subsection{Private patches}

As discussed in the section above, a developer may have various
changes to their local development repositories that they do not ever
wish to publish to a group repository (e.g. personal debugging code),
but which they would like to keep in their local repository, and
perhaps even share amongst their local repositories.

This is easily done via darcs, since those private changes can be
committed in patches that are separate from other patches; during the
process of pushing patches to the common repository ($R_c$), the
developer is queried for which patches should be moved to ($R_c$) on a
patch-by-patch basis.

The \verb!--complement! flag for the ``darcs pull'' operation can
further simplify this effort.  If the developer copies (via ``darcs
push'' or ``darcs pull'') all private patches into a special
repository/working tree ($R_p$), then those patches are easily
disregarded for pulling by adding \verb!--complement! to the ``darcs
pull'' line and listing $R_p$ after the primary source repository.

The \verb!--complement! flag is only available for ``darcs pull'', and
not ``darcs push'' or ``darcs send'', requiring the user to have pull
access to the target repository.  While the actual public repository
is often not shared in this manner, it's simple to create a local
version of the public repository to act as the staging area for that
public repository.

The following example extends the two feature addition example in the
previous section using a local staging repository ($R_l$) and a
private patch repository:

\begin{verbatim}
$ cd working-dir
$ darcs get http://server/repos/Rc Rl

$ darcs get Rl R1
$ cd R1
...development of f1
$ darcs record -m'p1: f1 initial work'
...
$ darcs record -m'p2: my debugging tracepoints'
...

$ cd ..
$ darcs get http://server/repos/Rc R2
$ cd R2
$ darcs pull -p p2 ../R1
... development of f2
$ darcs record -m'p3: f2 finished'

$ cd ..
$ darcs get Rl Rp
$ cd Rp
$ darcs pull -p p2 ../R2

$ cd ../Rl
$ darcs pull --complement ../R2 ../Rp
$ darcs send
... for publishing f2 patches to Rc

$ cd ../R1
$ darcs pull ../R2
... updates R1 with f2 changes from R2
... more development of f1
$ darcs record -m'p4: f1 feature finished.'

$ cd ../Rl
$ darcs pull --complement ../R1 ../Rp
$ darcs send
\end{verbatim}



\chapter{Repository formats}

Darcs 2 supports three repository formats:

\begin{itemize}
\item The current format, called `darcs-2'.  It has the most features.
\item The original format, called `darcs-1' or `old-fashioned inventory'.
  It is the only format supported by Darcs releases prior to 2.0.
  Patches cannot be shared between darcs-2 and darcs-1 repositories.
\item An intermediary format, called `hashed'.  It provides some of
  the features of darcs-2, while retaining the ability to exchange
  patches with darcs-1.  Patches cannot be shared between darcs-2 and
  hashed repositories.
\end{itemize}

Note that references to the hashed format refer to hashed darcs-1
repositories.  All darcs-2 repositories are also hashed, and `darcs
show repo' in a darcs-2 repository will report the format as `hashed,
darcs-2'.

The hashed (and darcs-2) format improves on the darcs-1 format in the
following ways:

\begin{itemize}
\item Improved atomicity of operations.  This improves the safety and
  efficiency of those operations.
\item File names in the `pristine' copy of the working tree are
  hashed.  This greatly reduces the chance of a program accidentally
  treating files in \_darcs/pristine as part of the working tree.

  For example, in darcs-1 running \verb|find -name "*.gif" -delete|
  instead of \verb|find -name _darcs -prune -o -name "*.gif" -delete|
  would result in GIF files in pristine files being deleted.

\item Support for `lazy' repositories.  This causes patches to be
  copied from the parent repository on demand, rather than during the
  initial `darcs get'.  For repositories that only operate on recent
  patches (such as feature branches), this reduces the repository's
  size, and the time and bandwidth taken to create it.

  Lazy repositories are first-class repositories, and all operations
  are supported as long as the parent repository remains accessible.
  This isn't the case for the `partial' feature of the darcs-1 format.

\item Support for caches that are shared by all repositories.  When
  operating on many similar repositories, the patches and pristine
  files they have in common are hard links to the cache.  This greatly
  reduces storage requirements, and the time and bandwidth needed to
  make similar repositories.
\end{itemize}

The darcs-2 format improves on the hashed format in the following
ways:

\begin{itemize}
\item The `exponential merge' problem is \emph{far} less likely to
  occur (it can still be produced in deliberately pathological cases).
\item Identical primitive changes no longer conflict.  For example, if
  two patches both attempt to add a directory `tests', these patches
  will not conflict.
\end{itemize}

% The following reference is particularly important because init
% explains *when* to use each format, which I don't want to
% copy-and-paste here because then the two copies will get out of
% sync.

See also `darcs initialize' (\ref{initialize}), `darcs convert'
(\ref{convert}) and `darcs get' (\ref{get}).


\chapter{Darcs commands}

\input{Darcs/Commands.lhs}

\section{Options apart from darcs commands}
\begin{options}
--help
\end{options}
Calling darcs with just \verb|--help| as an argument gives a brief
summary of what commands are available.
\begin{options}
--version, --exact-version
\end{options}
Calling darcs with the flag \verb|--version| tells you the version of
darcs you are using.  Calling darcs with the flag \verb|--exact-version|
gives the precise version of darcs, even if that version doesn't correspond
to a released version number.  This is helpful with bug reports, especially
when running with a ``latest'' version of darcs.
\begin{options}
--commands
\end{options}
Similarly calling darcs with only \verb|--commands| gives a simple list
of available commands.  This latter arrangement is primarily intended for
the use of command-line autocompletion facilities, as are available in
bash.

\section{Getting help}

\input{Darcs/Commands/Help.lhs}

\section{Creating repositories}

\input{Darcs/Commands/Init.lhs}

\input{Darcs/Commands/Get.lhs}

\input{Darcs/Commands/Put.lhs}

\section{Modifying the contents of a repository}

\input{Darcs/Commands/Add.lhs}

\input{Darcs/Commands/Remove.lhs}

\input{Darcs/Commands/Move.lhs}

\input{Darcs/Commands/Replace.lhs}

\section{Working with changes}

\input{Darcs/Commands/Record.lhs}

\input{Darcs/Commands/Pull.lhs}

\input{Darcs/Commands/Push.lhs}

\input{Darcs/Commands/Send.lhs}

\input{Darcs/Commands/Apply.lhs}

\section{Seeing what you've done}

\input{Darcs/Commands/WhatsNew.lhs}

\input{Darcs/Commands/Changes.lhs}

\input{Darcs/Commands/Show.lhs}

\section{More advanced commands}

\input{Darcs/Commands/Tag.lhs}

\input{Darcs/Commands/SetPref.lhs}

\input{Darcs/Commands/Check.lhs}

\input{Darcs/Commands/Optimize.lhs}

\section{Undoing, redoing and running in circles}

\input{Darcs/Commands/AmendRecord.lhs}

\input{Darcs/Commands/Rollback.lhs}

\input{Darcs/Commands/Unrecord.lhs}

\input{Darcs/Commands/Revert.lhs}

\input{Darcs/Commands/Unrevert.lhs}

\section{Advanced examination of the repository}

\input{Darcs/Commands/Diff.lhs}

\input{Darcs/Commands/Annotate.lhs}

% Includes the show commands.
\input{Darcs/Commands/Show.lhs}

\section{Rarely needed and obscure commands}

\input{Darcs/Commands/Convert.lhs}

\input{Darcs/Commands/MarkConflicts.lhs}

\input{Darcs/Commands/Dist.lhs}

\input{Darcs/Commands/TrackDown.lhs}

\input{Darcs/Commands/Repair.lhs}

\appendix

\chapter{Building darcs}
If your distribution provides a pre-built binary package of a recent
Darcs release, you are strongly encouraged to use that rather than
building Darcs yourself.

If you have (or can install) the \texttt{Haskell Platform}, this
is the next best option, as this will resolve build dependencies
automatically.  To download, compile and install Darcs and its
dependencies via \texttt{cabal-install}, the build tool provided by the
Haskell Platform, simply run
\begin{verbatim}
cabal update
cabal install darcs
\end{verbatim}

See \htmladdnormallinkfoot{http://haskell.org/platform/}{http://haskell.org/platform/}
for more information on the Haskell Platform.

\section{The Hard Way}
If you cannot install the Haskell Platform or \texttt{cabal-install}, you need to run
\begin{verbatim}
ghc --make Setup
./Setup configure
./Setup build
./Setup install
\end{verbatim}

This will require the following build dependencies:
\begin{itemize}
\item GHC 6.10 or higher; and
\item Cabal 1.6 or higher.
\end{itemize}

Additional build dependencies are declared in the \texttt{darcs.cabal}
file, and \texttt{./Setup configure} will tell you if any required
build dependencies aren't found.


\input{Darcs/Patch.lhs}

\input{Darcs/Repository/DarcsRepo.lhs}

\input{gpl.tex}

\end{document}


